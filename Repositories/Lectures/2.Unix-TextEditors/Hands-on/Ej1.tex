\documentclass[12pt]{article}
\usepackage[margin=0.5in]{geometry}
\usepackage{amsmath, amsthm}
\usepackage[spanish] {babel}


\title{\textsc{Editores de texto en UNIX}}

\begin{document}

\date{}
\maketitle


Usar lo visto en clase para hacer un script con el nombre de "susapellidos.sh" (en mi caso seria: \texttt{garavitocamargo.sh}).
El script debe hacer lo siguiente:\\

\begin{itemize}

\item Bajar el archivo \texttt{metamorphosis.txt} de:\\
\begin{scriptsize}
\verb"https://raw2.github.com/jngaravitoc/HerramientasComputacionales/master/2.Unix-TextEditors/Hands-on/metamorphosis.txt"
\end{scriptsize}  

\item Cuantes veces aparece cada una de las vocales en el archivo?

\item Bajar el archivo \texttt{Sainte-Beuve.txt} de: 

\begin{scriptsize}
\verb"https://raw2.github.com/jngaravitoc/HerramientasComputacionales/master/2.Unix-TextEditors/Hands-on/Sainte-Beuve.txt"
\end{scriptsize} 

\item Cual libro tiene mas veces la vocal \texttt{a} repetida y cuantas veces aparece?


\end{itemize}

\texttt{Tips:}\\

Como volver ejecutable un script: \texttt{chmod u+x garavitocamargo.sh}\\

Ejecutar el script: \texttt{./garavitocamargo.sh}\\

El comando \texttt{paste} sirve para pegar archivos como columnas i.e:\\

\texttt{cat data1.dat}\\
\texttt{> hola}\\

\texttt{cat data2.dat} \\
\texttt{> mundo}\\

\texttt{paste data1.dat data2.dat >\ data.dat}\\
\texttt{cat data.dat}\\
\texttt{> hola mundo}\\




Enviar el script a: \texttt{jn.garavito57} $[at]$ uniandes $[dot]$ edu $[dot]$ co  \\

Poner como asunto: ej1.HerramientasComputaciones y sus nombres.


\end{document}