\documentclass[12pt]{article}
\usepackage[margin=1.0in]{geometry}
\usepackage[utf8]{inputenc}
\usepackage[T1]{fontenc}
\usepackage{lmodern}
\usepackage[spanish]{babel}
\usepackage{amsmath}
\usepackage{graphicx}
\usepackage{multicol}

\title{Marcha Aleatoria}

\begin{document}
\date{}
\maketitle


\section*{Ejercicio:}

\textbf{Fecha de entrega: Jueves 24 de Abril 6pm}.

La tarea debe entregarse en un archivo comprimido que tenga como numbre sus apellidos \verb"garavito-camargo.zip"
y debe subirse al sicua, dentro de este archivo deben de ir los siguientes programas: 

\begin{itemize}


\item (25 puntos) Realizar un programa en python, en el cual una part\'icula realize una marcha aleatoria dentro 
   de una esfera de radio $\rm{R}$. La part\'icula esta en una posici\'on incial $(x=0, y=0, z=0)$ y se mueve en pasos aleat\'orios de distancia $1$ hasta llegar a una distancia $\rm{R}$ del centro. El programa debe escribir un archivo de datos (\verb"marcha-aleatoria.dat") con la siguiente informaci\'on: (x, y, z, Npasos, $r$) donde $r$ es las distancia al centro de la esfera. Este programa debe recibir por consola el radio $\rm{R}$ de la esfera (\verb"python marchaaleatoria.py 10").

\item (10 puntos) Realizar una gr\'afica del N\'umero de pasos vs la distancia ($r$) recorrida por la part\'icula.
	(Incluir labels, titulo.) Este programa debe leer los datos obtenidos en el punto anterior y debe correr as\'i:
	\verb"marcha-aleatoria.py marcha-aleatoria.dat" y debe realizar una grafica \verb"Npasos-r.png"
	
\item (15 puntos) Escribir un programa que haga un ajuste por m\'inimos cuadrados de la grafica anterior.
		este programa debe leer el archivo de datos \verb"marcha-aleatoria.dat" y debe retornar una gr\'afica 
		con el ajuste. la grafica debe incluir la ecuacion de la curva y debe llamarse \verb"ajuste.png".
		El programa de funcionar as\'i (\verb"python ajuste.py marcha-aleatoria.dat")

\item Bono (5 puntos) Hacer un grafica 3d en donde se muestre el movimiento de la part\'icula dentro de la esfera.
		Este progrmama debe leer el archivo \verb"marcha-aleatoria.dat" y debe retornar una grafica con el nombre \verb"3dplot.png" el programa debe funcionar as\'i (\verb"python plot3d.py marcha-aleatoria.dat")
\end{itemize}

\end{document}