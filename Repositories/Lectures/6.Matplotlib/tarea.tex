\documentclass[12pt]{article}
\usepackage[margin=1.0in]{geometry}
\usepackage[utf8]{inputenc}
\usepackage[T1]{fontenc}
\usepackage{lmodern}
\usepackage[spanish]{babel}
\usepackage{amsmath}
\usepackage{graphicx}
\usepackage{multicol}

\title{Tarea}

\begin{document}
\maketitle

1. En un notebook de python graficar las columnas $3$ y $4$ del archivo \verb"data.txt" (scatter), enfocar (cambiar los l\'imites de la gr\'afica) en la regi\'on en la que se ven mas puntos. Poner labels, title, y modifiquen la 
imagen a su gusto.\\


Ayuda: Para subir los datos al notebook usen \verb"np.loadtxt('data.txt')"

\textbf{Fecha de entrega: Lunes 10 de marzo a media noche}

\end{document}