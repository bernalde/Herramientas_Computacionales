\documentclass[12pt]{article}
\title{Notas de APO II}
\usepackage[margin=1.0in]{geometry}
\usepackage[utf8]{inputenc}
\usepackage[T1]{fontenc}
\usepackage{lmodern}
\usepackage[spanish]{babel}

\begin{document}
 \maketitle
\section*{NIVEL 7}
\begin{itemize}
 \item Algoritmos de bùsqueda\\
 \item Algoritmos de ordenamiento\\
 
 - Burbuja (intercambio)
   
   1. Tomar un elemento (primero)
   
   2. Comparar el elemento con el siguente\\
     - Si estàn ordenados no se hace nada\\
     - Si estàn desordenados se intercambian\\

   3. Repetir hasta llegar al ùltimo

   4. Repetir tantas veces como elementos existan

   

   Ejemplo:
  
   for (int i = animales.size(); i>0; i- -)\\
   {\\
    for (int j = 0; j<i-1; j++)\\
    {\\
     Animal a1 = (Animal) animales.get(j);\\
     Animal a2 = (Animal) animales.get(j+1);\\
     
     if (a1.compararPorPeso(a2)>0)\\
     {\\
      animales.set(j,a2);\\
      animales.set(j+1,a1);\\
     }\\
    }\\
   }\\


  - Selecciòn

    1. Buscar el menor de la parte desordenada

    2. se intercambia el menor con el primero de la parte desordenada

    3. Repetir hasta acabar con todos los elementos



    Ejemplo:

    for (int i = 0; i<animales.size()-1; i++)\\
    {\\
     int posMenor = i;\\
     Animal menor = (Animal) animales.get(i);\\

     for (int j = i+ 1; j<animales.size(); j++)\\
     {\\
      Animal a1 = (Animal) animales.get(j);\\
      
      if (a1.compararPorPeso(menor) <0)\\
      {\\
       menor = a1;\\
       posMenor = j;\\
      }
     }
     Animal prim = (Animal) animales.get(i);\\
     animales.set(i,menor);\\
     animales.set(posMenor,prim);\\
    }

     


 \item Pruebas unitarias\\

 \item El invariante de una clase\\
 - Nunca cambia
  
 - Las propiedades de una clase que nunca deben variar
 
 - Se define sobre los atributos de la clase
 
 - Se verifica en los mètodos que modifican datos (incluyendo el constructor)

\end{itemize}






\end{document}
