\documentclass{article}
\usepackage[margin=3.0cm]{geometry}
\usepackage{amsmath}
\title{ Notas Mecanica de Materiales}

\begin{document}

\begin{huge}
 Notas Mec\'anica de Materiales
\end{huge}

Laura Catalina C\'aceres Torres

\section{Transformaci\'on de Esfuerzos}

Esfuerzo Plano:  Estado de esfuerzo en el cual el esfuerzo normal en el eje  de Z, perpendicular al plano  x-y y todos los esfuerzos cortantes asociados perpendiculares 
al plano x-y son asumidos como de magnitud 0.

\subsection{Ecuaciones Generales de la transformaci\'on de esfuerzo plano}



\begin{equation}
 \Sigma F_{X}= 0:   \sigma_{\dot{x}}=\frac{\sigma_{x}+ \sigma_{y}}{2} + \frac{\sigma_{x} -\sigma_{y}}{2}  cos2\theta + \tau_{xy}sen2\theta 
\end{equation}



\begin{equation}
  \Sigma F_{X}= 0:   \tau_{\dot{xy}}= - \frac{\sigma_{x}-\sigma_{y}}{2}  sen2\theta + \tau_{xy}cos2\theta 
\end{equation}

entonces 



\begin{equation} 
\sigma_{\dot{y}} = \frac{\sigma_{x}+ \sigma_{y}}{2}-\frac{\sigma_{x}-\sigma_{y}}{2} cos2\theta -\tau_{xy}sen2\theta
\end{equation}

donde se reemplaza 


\begin{equation} 
\theta = \theta + 90 en (1)
\end{equation}



\section{Esfuerzos Principales Axiales y Maximo de Corte}

Como puedo encontrar el plano que me produce los esfuerzos normales maximos y minimos?


\begin{equation}
   \sigma_{\dot{x}}=\frac{\sigma_{x}+ \sigma_{y}}{2} + \frac{\sigma_{x} -\sigma_{y}}{2}  cos2\theta + \tau_{xy}sen2\theta
\end{equation}



\begin{equation}
 \frac{d\sigma_{x}}{d\theta} = - \frac{\sigma_{x}+ \sigma_{y}}{2} 2sen2\theta + 2\tau_{xy}cos2\theta  = 0
\end{equation}

 resolviendo


\begin{equation}
Tan2\theta_{p}=\frac{\tau_{xy}}{\sigma_{x}+\sigma_{y}/2}
\end{equation}



\section{Circulo de Morh para Esfuerzo Plano}


En dos dimensiones, la Circunferencia de Mohr permite determinar la tensi\'on m\'axima  y m\'inima, a partir de dos mediciones de la tensi\'on normal
 y tangencial sobre  dos \'angulos que forman 90grados. 


Es importante recordar que el ejevertical se encuentra invertido, por lo que esfuerzos positivos van hacia abajo y esfuerzos 
negativos se ubican en laparte superior.El eje horizontal presenta la tension normal (sigma) y el eje vertical representa la tension cortante o tangencial (tao


\subsection{Centro del Circulo de Mohr}

\begin{equation}
C= ( \sigma_{prom} , 0) = \frac{\sigma_{x}+ \sigma_{y}}{2}
\end{equation}

\subsection{Radio de la Circunferencia del Cicurlo de Mohr}

\begin{equation}
 R=\sqrt{\frac{(\sigma_{x}+\sigma_{y}}{2})^2 +{\tau_{xy}}^2}
\end{equation}


\subsection{Pasos para la construcci\'o del C\'iculo de Mohr para esfuerzo plano}

1.Dibuje un eje de coordenadas como SIGMA como abscisa (positivo hacia la derecha) y TAO como ordenada(positivo hacia abajo).

2.Localice el centro C del ciculo en el punto como coordenadas ( \sigma_{prom} , 0).

3.Localizar el Punto A (Represente las condiciones de esfuerzo sobre la cara A del elemento) con coordenadas (\sigma_{x} , \tau_{xy}).

4.Localizar el punto B (representa las condiciones de esfuerzo sobre la cara B del elemento) se localiza diametral con respecto a A (\sigma_{y},-{\tau_{xy})

5. Con C como centro, trace el circulo de Mohr por los puntos A y B. 

\end{document}
