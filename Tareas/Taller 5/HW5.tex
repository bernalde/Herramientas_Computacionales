%--------------------------------------------------------------------
%--------------------------------------------------------------------
% Formato para los talleres del curso de Herramientas Computacionales
% Universidad de los Andes
%--------------------------------------------------------------------
%--------------------------------------------------------------------

\documentclass[11pt,letterpaper]{exam}
\usepackage[utf8]{inputenc}
\usepackage[spanish]{babel}
\usepackage{graphicx}
\usepackage{mdframed}
\usepackage{tabularx}
\usepackage[absolute]{textpos} % Para poner una imagen completa en la portada
\usepackage{multirow}
\mdfdefinestyle{mystyle}{leftmargin=1cm,rightmargin=1cm,linecolor=red}
%\usepackage{pst-barcode}
%\usepackage{auto-pst-pdf}

\newcommand{\base}[1]{\underline{\hspace{#1}}}
\boxedpoints
\pointname{ pt}
%\extrawidth{0.75in}
%\extrafootheight{-0.5in}
\extraheadheight{-0.15in}
%\pagestyle{head}

%\noprintanswers
%\printanswers
\renewcommand{\solutiontitle}{}
\SolutionEmphasis{\color{blue}}

\usepackage{upquote,textcomp}
\newcommand\upquote[1]{\textquotesingle#1\textquotesingle} % To fix straight quotes in verbatim

\begin{document}
\begin{center}
{\Large Herramientas Computacionales} \\
Taller 5- \textsc{Python} \\
{\small \it Septiembre 2 de 2014}
\end{center}

%\begin{textblock*}{40mm}(10mm,20mm)
%  \includegraphics[width=3cm]{logoUniandes.pdf}
%\end{textblock*}

%\begin{textblock*}{40mm}(161mm,20mm)
%  \includegraphics[width=3cm]{logoUniandes.pdf}
%\end{textblock*}

\vspace{1cm}

Las respuestas a los ejercicios deben ser entregadas en dos archvos : \verb+NombreApellido_HW5-1.py,+ y \verb+NombreApellido_HW5-2.py+\\

En cada ejercicio de este taller se otorga $1/3$ de los puntos correspondientes si el código propuesto es razonable, $1/3$ si se puede ejecutar y $1/3$ si entrega resultados correctos.

\begin{questions}

\question[50] Escriba un programa en \verb+Python+ usando {\bf funciones} que juege "Piedra, papel y tijera", el programa debe como recibir como argumento 
la eleccion del jugador. En el caso en el que el jugador no escoja el programa debe imprimir un mensaje indicando como debe ejecutarse el juego. El programa
debe imprimir la decision del jugador, la del computador y despues debe decir quien fue el ganador entre el jugador y el computador. Las siguientes funciones deben incluirse en el programa:

\begin{itemize}
\item Una funci\'on que asigne un n\'umero tanto como a Piedra a Papel y Tijera.
\item Una funci\'on que asinge Piedra, Papel o Tijera a un n\'umero.
\item Una funci\'on que compare quien es el ganador. 
\end{itemize}



\question[50] Usando recurrencia realizar un programa en python que realice la serie de 'Tribonacci' es decir una serie que empieze con 1 1 1 y cada n\'umro siguiente sea la suma de los tres anteriores. La seria debe ser as\'i:

\verb+[1, 1, 1, 3, 5, 9, 17, 31, 57, 105, 193, 355, 653, 1201, 2209, 4063, 7473, 13745, 25281]+\\

El n\'umero de terminos  de la serie puede ser modificado por el usurario.
%{\bf Learnig to Mining the Data}) En el archivo \verb"Serena-Venus.txt" esta la informaci\'on de una simulacion de galaxias. Cada punto en esta simulaci\'on corresponde a una galaxia. Cada columna corresponde al: ID de la galaxia, Posicion en x, Posicion en y, Posicion en z, Velocidad en x, Velocidad en y, Velocidad en z y Masa.  Escribir un progrma en \verb+python+ que lea estos datos y calcule la Fuerza gravitacional que siente cada una de las galaxias debida a todo el campo de galaxias. 

%begin{itemize}
%item ¿Cuales son las diez galaxias que sienten mas fuerza gravitacional y las diez que menos? Para esto el programa debe escribir un archivo \verb+FuerzasMayores.txt+ y \verb+FuerzasMenores.txt+ en el cual este el ID de la galaxia y el valor de la Fuerza Gravitacional.
%end{itemize}

%\bf Bono y Anitbono:} El programa que corra en menos de 3.5 min recibe 20 puntos de bono. Los que corran en menos de 4.5 min 10  puntos de bono. Los que tarden mas de 6 min se le restan 10 puntos y los que tarden mas de 10 min se le quitanr\'an 20 puntos.

\end{questions}
\end{document}
